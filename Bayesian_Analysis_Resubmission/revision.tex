\documentclass[ba]{imsart}
%
\pubyear{0000}
\volume{00}
\issue{0}
\doi{0000}
%\arxiv{}
\firstpage{1}
\lastpage{1}

\usepackage{adjustbox}
\usepackage{placeins}
\usepackage{mathrsfs}
\usepackage{subfigure}
\usepackage{amsthm}
\usepackage{amsmath}
\usepackage{amsfonts}
\usepackage{natbib}
\usepackage[colorlinks,citecolor=blue,urlcolor=blue,filecolor=blue,backref=page]{hyperref}
\usepackage{graphicx}

\startlocaldefs
% ** Local definitions **
\endlocaldefs

\begin{document}

%% *** Frontmatter *** 

\begin{frontmatter}
\title{Bayesian Inference for Cox Proportional Hazard Models with Partial Likelihoods, Semi-Parametric Covariate Effects and Correlated Observations}

%\title{\thanksref{T1}}
%\thankstext{T1}{<thanks text>}
\runtitle{}

\begin{aug}
%\author{\fnms{} \snm{}}
%\author{\fnms{<firstname>} \snm{<surname>}\thanksref{}\ead[label=e1]{}}
%\and
%\author{\fnms{} \snm{}}
\author{\fnms{Ziang} \snm{Zhang}},
\author{\fnms{Alex} \snm{Stringer}},
\author{\fnms{Patrick} \snm{Brown}}
\and
\author{\fnms{James} \snm{Stafford}}%


\runauthor{Zhang et al.}

\address[addr1]{}


%\thankstext{<id>}{<text>}

\end{aug}

\begin{abstract}
We introduce a novel approximate Bayesian inference methodology for the Cox Proportional Hazards model with partial likelihood that allows the inclusion of semi-parametric covariate effects and correlated survival times. We use quasi-newton optimization to improve computation in the presence of a dense log likelihood Hessian matrix, in contrast with existing methods for Bayesian inference in similar models which require this to be sparse and hence cannot be used with partial likelihood. We further improve on existing methods by using an adaptive quadrature technique to reduce the amount of specialist user input required to fit the model, and to minimize the number of dense Hessian matrices required to be stored. A simulation study shows that our proposed method provides accurate inference in a variety of settings. We demonstrate the practical utility of our method through the analysis of Leukaemia survival times, with a semi-parametric covariate effect, and Kidney infection times, which are paired. An R package implementing our method will be released publicly.
\end{abstract}

%% ** Keywords **
\begin{keyword}
\kwd{Cox Proportional Hazard Model}
\kwd{Partial Likelihood}
\kwd{Approximate Bayesian inference}
\kwd{Semi-parametric Smoothing}
\end{keyword}

\end{frontmatter}

%% ** Mainmatter **

%\section{}\label{}

% \begin{figure} 
% \includegraphics{<eps-file>}% place <eps-file> in ./img  subfolder
% \caption{}
% \label{}
% \end{figure}


% \begin{table} 
% *****************
% \begin{tabular}{lll}
% \end{tabular}
% *****************
% \caption{}
% \label{}
% \end{figure}

%%%%%%%%%%%%%%%%%%%%%%%%%%%%%%%%%%%%%%%%%%%%%%
%% Supplementary Material, if any, should   %%
%% be provided in {supplement} environment  %%
%% with title and short description.        %%
%%%%%%%%%%%%%%%%%%%%%%%%%%%%%%%%%%%%%%%%%%%%%%
%\begin{supplement}
%\stitle{???}
%\sdescription{???.}
%\end{supplement}

%% ** The bibliograhy **
%\bibliographystyle{ba}
%\bibliography{<bib-data-file>}% place <bib-data-file> in ./bib folder 

% ** Acknowledgements **
% \begin{acknowledgement}
% \end{acknowledgement}

\section{Introduction}\label{sec1}
For problems involving time-to-event data, the combination of Cox proportional hazard (Cox PH) models and inference via partial likelihood has been the dominant methodology following its development by Cox \citep{coxph}. The Cox PH model assumes that any two subjects' event hazards are proportional as a function of time, with the ratio depending on unknown covariate effects which are inferred from the observed data. Event times may be correlated within the sample, for example when the response is time to kidney failure for the left and right kidneys from the same subject. Inference that is conducted via partial likelihood does not require assumptions to be made about the form of the baseline hazard. Further, the use of Bayesian inference with the Cox PH model is desirable as this yields model-based estimation and uncertainty quantification for all parameters of interest in the presence of complex models for the hazard, which would be difficult to achieve otherwise. However, existing methods for approximate Bayesian inference based on Integrated Nested Laplace Approximations (INLA) \citep{inla} cannot be applied to the Cox PH model with partial likelihood because the Hessian matrix of the log partial-likelihood is fully dense while INLA requires this matrix to be diagonal. Application of the INLA methodology to the Cox PH model without partial likelihood has been considered \citep{inlacoxph}, but this requires smoothness assumptions to be made about the baseline hazard.

Recently, \cite{casecross} developed an approximate Bayesian inference methodology for case-crossover models, which applies the approximation strategy of INLA to a log-partial likelihood with a non-diagonal Hessian matrix. Their methodology includes semi-parametric covariate effects and yields full posterior uncertainty for the corresponding smoothness parameters, an improvement over existing frequentist methods. Though related, the partial likelihood they consider is simpler than that of the Cox PH model, and the Hessian matrix of their log-partial likelihood is block-diagonal and sparse. In contrast, the Hessian matrix of log-partial likelihood of Cox PH model is fully dense, so the method of \cite{casecross} does not apply to this model. Further, they use a manual integration strategy which requires the user to supply their own grid, a tedious operation which requires specialist knowledge to do properly. This limits the practical utility of their method.

In this paper we extend the approximate Bayesian inference methodologies of \cite{casecross} and \cite{inlacoxph} to the Cox proportional hazard model with partial likelihood. Our methodology accommodates semi-parametric smoothing effects and correlation between observed survival times, which we demonstrate through a simulation study and the analysis of two datasets. To accomodate the fitting of this more difficult model, we improve upon the computations of \cite{casecross} in three ways: by using a low-rank approximation to the (dense) Hessian matrix within the optimization over the parameter space; by modifying their prior precision matrix to be more sparse; and by using adaptive quadrature to choose the integration grid.

The remainder of this paper is organized as follows. In \S\ref{sec:model} we describe the semi-parametric Cox PH model and the existing approach to approximate Bayesian inference. In \S\ref{sec:method}, we describe our novel methodology and our three improvements introduced to mitigate the computational challenges presented by the complicated partial likelihood. In \S\ref{sec:example} we illustrate our methodology in a simulation study and through the analysis of Leukaemia survival data analysed by \cite{inlacoxph} and the Kidney catheter data analysed by \cite{kidney}. We conclude in \S\ref{sec:discussion} with a discussion.

\section{Model}\label{sec:model}

\subsection{A latent Gaussian Cox PH Model}

Suppose we observe $n$ groups indexed by $i$, each with $n_{i}$ observations indexed by $j$. For example, we may observe $n$ subjects with $n_{i}$ measurements per subject. Denote the random variable representing the $j^{th}$ survival time in the $i^{th}$ group by $T_{ij}$, and denote its realization by $t_{ij}$. Let $c_{ij}$ denote the censoring time for observation $T_{ij}$ such that $T_{ij}$ is not directly observable when $c_{ij} < T_{ij}$. The observed survival time is $y_{ij} = \min\{t_{ij},c_{ij}\}$. Define $d_{ij} = 1$ if $y_{ij} = t_{ij}$ (a survival time) and $d_{ij} = 0$ if $t_{ij} > y_{ij}$ (a censoring time). The observations for each $i,j$ are hence denoted by pairs $y =  \left\{(y_{ij},d_{ij}): i = 1,\ldots,n; j = 1,\ldots,n_{i} \right\}$. The total number of rows in the data set is denoted by $N = \sum_{i=1}^{n}n_{i}$.

Define $h_{ij}(t)$ to be the hazard function for the random variable $T_{ij}$. The Cox PH model assumes $h_{ij}(t) = h_0(t)\text{exp}(\eta_{ij})$ where $h_0(t)$ is an unknown baseline hazard function that does not depend on the covariates. An additive predictor $\eta_{ij}$ links the covariates for the $ij$th observation to the survival time $T_{ij}$:
\begin{equation}\begin{aligned}\label{eqn:eta}
\eta_{ij} &=x_{ij}^{T}\beta+\sum_{q=1}^{r} \gamma_q(u_{qij}) +\xi_{i} \\
\xi_i | \sigma_{\xi} &\overset{iid}{\sim} \mathcal{N}(0,\sigma_{\xi}) \\
\gamma_{q}(\cdot)|\sigma_{q} &\overset{ind}{\sim} \mathcal{GP}\left(0,\mathcal{C}_{\sigma_q}\right), q = 1,\ldots,r
\end{aligned}\end{equation}
Let $\eta = \left\{ \eta_{ij}: i = 1,\ldots,n; j = 1,\ldots,n_{i}\right\}$ be the vector of all the additive linear predictors. Here $x_{ij}$ is a $p$-dimensional vector of covariates that are modelled as having linear associations with the log-hazard, and $\beta = (\beta_{1},\ldots,\beta_{p})$ are regression coefficients. The $u_{q} = \left\{u_{qij}: i = 1,\ldots,n; j = 1,\ldots,n_{i} \right\}, q = 1,\ldots,r$ are covariate vectors whose association with the log-hazard is modelled semi-parametrically through unknown smooth functions $\gamma_1,\ldots,\gamma_r$. The vector of group intercepts $\xi = \left\{ \xi_{i}: i=1,\ldots,n\right\}$, referred to as ``frailty'' coefficients in the context of survival analysis \citep{frailty}, are included to model correlation between survival times coming from the same group $i$. There is no global intercept $\beta_{0}$ as this would be absorbed by $h_{0}(t)$.

\subsection{Modelling Semi-parametric covariate effect}\label{subsec:smooth}
The semi-parametric covariate effect $\gamma_q$ are modelled as independent zero-mean Gaussian processes defined by their covariance functions $C_{\sigma_{q}}$ which in turn parametrized by $\sigma_q > 0$. A typical choice
of covariance function is the Random Walk of order 2 (RW2, \citet{rw2}), which has a connection to cubic smoothing splines.

Let $U_{q} = \{U_{ql};l = 1, ...., m_q\}$ be the ordered vector of distinct values of covariate $u_q,q = 1,\ldots,r$; often these values are set by the user by discretizing the covariate $u_q$ into $m_q$ pre-specifed bins. To infer the infinite-dimensional parameters $\gamma_{q},q = 1,\ldots,r$, we approximate each $\gamma_q$ by a piecewise constant function with jumps at the $U_{ql}$, which we denote as $\gamma_{q}(U_{ql}) = \Gamma_{ql}$. We define the vectors of function values $\Gamma_{q} = \left\{ \Gamma_{q1},\ldots,\Gamma_{qm_{q}}\right\}$ having distributions $\Gamma_{q}|\sigma_{q}\sim\mathcal{N}\left[ 0,\Sigma_{q}(\sigma_{q})\right]$ for each $q = 1,\ldots,m_{q}$. These distributions are parametrized through their precision matrices $\Sigma_{q}^{-1}(\sigma_{q})$ corresponding to the specific Gaussian processes chosen, which depend on parameters $\sigma_{q}$. We define $\Gamma = (\Gamma_{1},\ldots,\Gamma_{r})$ and write $\Gamma|\sigma_{1},\ldots,\sigma_{q}\sim\mathcal{N}\left( 0,\Sigma^{-1}_{\Gamma}\right)$ with $\Sigma^{-1}_{\Gamma} = \text{diag}\left[ \Sigma_{1}^{-1}(\sigma_{1}),\ldots,\Sigma_{r}^{-1}(\sigma_{r})\right]$. These precision matrices are available in closed form and no large matrices need to be inverted to compute them in the present application \citep{rw2}.

These models usually contain an intercept $\beta_{0}$ and a \emph{sum-to-zero} constraint $\sum_{l=1}^{m_q}\Gamma_{ql} = 0$, for identifiability of parameters. However, $\beta_{0}$ is not identifiable when using the partial likelihood for inference, and hence the sum-to-zero constraint is difficult to interpret in this setting. We fit the following modified RW2 model for each $q = 1,\ldots,r$:
\begin{equation}\begin{aligned}\label{eqn:rw2}
\Gamma_{q,l+1} - 2\Gamma_{q,l} + \Gamma_{q,l-1} &\overset{iid}{\sim}\text{N}\left( 0,\sigma^{2}_{q}\right), \\
\Gamma_{q,a} = 0,
\end{aligned}\end{equation}
where $a\in\left\lbrace 1,\ldots,m_{q}\right\rbrace$ is some chosen reference value. This parametrization is identifiable under the partial likelihood and gives a clear interpretation of $\Gamma_{q,l}$ as the change in log-risk for an individual with covariate value $u_{q,l}$ compared to an individual with covariate value $u_{q,a}$. 

Finally, define the variance parameter vector $\theta = (\theta_{0},\ldots,\theta_{r})$ where $\theta_{q} = -2\log\sigma_{q},q = 1,\ldots,r$, and $\theta_{0} = -2\log\sigma_{\xi}$. The variance parameters are given prior distribution $\theta \sim \pi(\theta)$. 

\subsection{Approximate Bayesian Inference}

Inference is carried out via a partial likelihood function. Define the \textit{risk set} $R_{ij} = \left\{k,l : y_{kl} \geq y_{ij}\right\}$. Assuming $y_{ij} \neq y_{kl}$ when $(i,j) \neq (k,l)$, the partial likelihood can be written as follows: 
\begin{equation}\begin{aligned}\label{eqn:partial}
\pi(y|\eta) &= \prod_{i=1}^{n}\prod_{j=1}^{n_{i}} \bigg\{\frac{\exp[\eta_{ij}]}{{\sum_{l,k\in R_{ij}}^{}\exp[\eta_{lk}]}}\bigg \}^{d_{ij}} \\
&= \prod_{i=1}^{n}\prod_{j=1}^{n_{i}} \bigg\{\frac{1}{{1 + \sum_{l,k\in R_{ij} , (l,k) \neq (i,j)}\exp[\Delta_{lk,ij}]}}\bigg \}^{d_{ij}} \\
\end{aligned}\end{equation}
where $\Delta_{lk,ij} = \eta_{lk} - \eta_{ij}$. Note that $h_{0}(t)$ does not appear in the partial likelihood, and hence inference may be carried out in the absence of assumptions about $h_{0}(t)$. 

The partial likelihood (\ref{eqn:partial}) can be written in the following form:
\begin{equation}\begin{aligned}\label{eqn:whyINLAfail1}
\pi(y|\eta) &= \prod_{i=1}^{n}\prod_{j=1}^{n_{i}} \pi(y_{ij}|\eta),
\end{aligned}\end{equation}
while in order for a model to be compatible with INLA, its likelihood must have the form:
\begin{equation}\begin{aligned}\label{eqn:whyINLAfail2}
\pi(y|\eta) &= \prod_{i=1}^{n}\prod_{j=1}^{n_{i}} \pi(y_{ji}|\eta_{ij}).
\end{aligned}\end{equation}
\cite{casecross} extend this to permit partial likelihoods of the form:
\begin{equation}\begin{aligned}\label{eqn:casecrosslik}
\pi(y|\eta) &= \prod_{i=1}^{n}\prod_{j=1}^{n_{i}} \pi(y_{ji}|\eta_{i}).
\end{aligned}\end{equation}
which still does not include (\ref{eqn:partial}). \cite{inlacoxph} are able to write the likelihood for their Cox PH model in the form (\ref{eqn:whyINLAfail2}) using the full, not partial likelihood (\ref{eqn:partial}). Because of this, they require assumptions to be made about the baseline hazard.

Further define $\Delta_{lk,ij} = \eta_{lk} - \eta_{ij}$ in terms of the additive predictors (\ref{eqn:eta}). Note that $\Delta_{lk,ij} = \Delta_{11,ij} - \Delta_{11,lk}$ for every $(i,j,l,k)$. To simplify notation, define $\Delta_{ij} = \Delta_{11,ij}$, and note that $\Delta_{11} = 0$. The entire partial likelihood (\ref{eqn:partial}) depends on $\eta$ only through  $\Delta = \left\{\Delta_{ij}: i = 1,\ldots,n; j = 1,\ldots,n_{i} \right\}$. For the remainder of the paper we reflect this in our notation, writing $\pi(y|\Delta) \equiv \pi(y|\eta)$ and defining the log-likelihood $\ell(\Delta; y) = \log\pi(y|\Delta)$.

Define $W = \left(\Delta, \Gamma,\beta, \xi \right)$ which we refer to as the \textit{latent parameters} and let $\text{dim}(W) = m$. The RW2 prior for $\gamma$ is improper, and the precision matrix $\Sigma_{\Gamma}$ is singular. Approximate Bayesian inference of the type we consider requires the precision matrix of $W$ to be nonsingular \citep{inla,inlacoxph,casecross}. We follow these sources and introduce a small noise term $\epsilon_{ij} \stackrel{iid}{\sim} \text{N}(0,\tau^{-1})$ (for some large, fixed $\tau$) into the model to make the required matrices nonsingular. However, in contrast to \citet{casecross}, we redefine:
\begin{equation}
\Delta_{ij} = \eta_{11} - \eta_{ij} + \epsilon_{ij},
\end{equation}
adding the noise onto the \emph{differenced} additive predictor. It will be shown in \S\ref{sec:method} that this improves computation by improving the sparsity of large matrices involved in the required calculations. The addition of these $\epsilon_{ij}$ gives the joint distribution of $\left(\Delta, \Gamma,\beta, \xi \right)$ a non-singular precision matrix, and enables the use of improper prior in the model specification. We set $\tau = \exp(7)$ which is well within the broad range of $\exp(2),\ldots,\exp(14)$ which \cite{casecross} found to yield very similar inferences and running times. 

Our model specifies $W|\theta\sim\text{N}\left[ 0,Q^{-1}_{\theta}\right]$. An expression for $Q_{\theta}$ is given in \S\ref{sec:method} and a derivation is given in Appendix A. Our main inferential interest is to obtain the marginal posterior distributions of the latent parameters:
\begin{equation}\begin{aligned}\label{eqn:interestedQuat3}
\pi(W_{s}|y) = \int \pi(W_{s}|y,\theta) \pi(\theta|y) d\theta, s = 1,\ldots,m  \\
\end{aligned}\end{equation}
These are used for point estimates and uncertainty quantification of the latent parameters, which often include the effects of primary interest. We are also interested in the joint posterior distributions of the variance parameters:
\begin{equation}\begin{aligned}\label{eqn:interestedQuat1}
\pi(\theta|y) = \frac{\int \pi(W,y,\theta) dW}{\int_{} \int_{} \pi(W,y,\theta) dW d\theta } \\
\end{aligned}\end{equation}
These are used for point estimates and uncertainty quantification of the variance parameter $\theta$, and appear as integration weights in (\ref{eqn:interestedQuat3}). Of secondary inference is the joint posterior distribution of the latent parameters:
\begin{equation}\begin{aligned}\label{eqn:interestedQuat2}
\pi(W|y) = \int \pi(W|y,\theta) \pi(\theta|y) d\theta  \\
\end{aligned}\end{equation}
This appears primarily as an intermediate step in the calculation of the marginal posteriors (\ref{eqn:interestedQuat3}).

All of the quantities of interest (\ref{eqn:interestedQuat3}) -- (\ref{eqn:interestedQuat2}) depend on intractable high-dimensional integrals. \cite{casecross} utilize Gaussian and Laplace approximations combined with numerical quadrature to approximate each of these integrals accurately and efficiently. Their approximations take the form:
\begin{equation}\begin{aligned}\label{eqn:integration}
\tilde{\pi}(W_{s}|y) &= \sum_{k=1}^{K}
\tilde{\pi}_{G}(W_{s}|y,\theta^{k})
\tilde{\pi}_{LA}(\theta^{k}|y)\delta_{k}, s = 1,\ldots,m \\
\tilde{\pi}(W|y) &= \sum_{k=1}^{K}
\tilde{\pi}_{G}(W|y,\theta^{k})
\tilde{\pi}_{LA}(\theta^{k}|y)\delta_{k} \\
\end{aligned}\end{equation}
where $\left\{\theta^{k},\delta_{k}\right\}_{k=1}^{K}$ is a set of nodes and weights corresponding to a manually-rescaled Gauss-Hermite quadrature rule. The $\tilde{\pi}_{G}(W_{s}|y,\theta^{k})$ is a Gaussian approximation for $\pi(W_{s}|y,\theta^{k})$ and the $\tilde{\pi}_{LA}(\theta^{k}|y)$ is a Laplace approximation for $\pi(\theta^{k}|y)$, which we describe at below.

The approximations (\ref{eqn:integration}) are computed as follows. For any fixed $\theta$, define
\begin{equation}\begin{aligned}\label{eqn:modeandhessian}
\widehat{W}_{\theta} = \left( \widehat{\Delta}_{\theta},\widehat{\Gamma}_{\theta},\widehat{\beta},\widehat{\xi}_{\theta}\right) &= \text{argmax}_{W}\log\pi(W|\theta,y) \\ 
H_{\theta}(W) &= -\frac{\partial^{2}}{\partial W \partial W^{T}}\log\pi(W|\theta,y) \\
v_{\theta,s}^{2} &= \left[H_\theta \left(\widehat{W}_{\theta}\right) ^ {-1} \right]_{ss}, s = 1,\ldots,m
\end{aligned}\end{equation}
For the conditional posterior
\begin{equation}\begin{aligned}\label{eqn:condpost}
\pi(W|\theta,y) \propto \exp\left\lbrace -\frac{1}{2}W^{T}Q_{\theta}W + \ell\left(\Delta;Y\right)\right\rbrace,
\end{aligned}\end{equation}
a second-order Taylor expansion of $\log\pi(W|\theta,y)$ about $W = \widehat{W}_{\theta}$ yields a Gaussian approximation:
\begin{equation}\label{eqn:gaussianapprox}
\pi(W|\theta,y) \approx \tilde{\pi}_{G}(W|y,\theta) \propto \text{exp}\left\{-\frac{1}{2} \left(W-\widehat{W}_{\theta} \right)^T H_\theta\left(\widehat{W}_{\theta}\right) \left(W-\widehat{W}_{\theta} \right) \right\} \\
\end{equation}
Direct integration of this Gaussian approximation yields a Gaussian approximation for the corresponding marginal density:
\begin{equation}\label{eqb:marginalgaussianapprox}
\tilde{\pi}_{G}(W_{s}|y,\theta) = \int\tilde{\pi}_{G}(W|y,\theta)dW_{-s} \propto\text{exp}\left\{-\frac{1}{2v_{\theta,s}^{2}} \left(W_s-\widehat{W}_{\theta s} \right)^2 \right\}, s = 1,\ldots,m
\end{equation}
For the joint posterior of the variance parameters, the method of \cite{tierney} yields a Laplace approximation:
\begin{equation}\begin{aligned}\label{eqn:laplace}
\pi(\theta|y) \approx \tilde{\pi}_{LA}(\theta|y) \propto \pi(\theta)\left\{\frac{\left|Q_{\theta}\right|}{\left|H_{\theta}\left(\widehat{W}_{\theta}\right)\right|}\right\}^{1/2}\exp\left\{ -\frac{1}{2}\widehat{W}_{\theta}^{T}Q_{\theta}\widehat{W}_{\theta} + \ell\left(\widehat{\Delta}_{\theta};y \right)\right\}
\end{aligned}\end{equation}
With these approximations available, inference for $W$ makes use of the approximation (\ref{eqn:integration}).

Computing the approximations (\ref{eqn:integration}) requires choosing a quadrature rule consisting of nodes $\left\{\theta^{k}\right\}_{k=1}^{K}$ and weights $\left\{\delta_{k}\right\}_{k=1}^{K}$ for some chosen $K\in\mathbb{N}$. \cite{casecross} lay a user-chosen grid over a range of $\theta$ that is thought to be plausible, and then compute the Gaussian (\ref{eqn:gaussianapprox}) and Laplace (\ref{eqn:laplace}) approximations at each point on this grid. This requires the user to choose the location and spread of the grid points, as well as a number $K$
of points that is large enough such that the structure of the resulting posterior approximations is captured. The function $\pi(W|Y,\theta)$ must be optimized, and the Hessian matrix stored, for each of these $K$ points. In addition to this strategy requiring the user to have specialist knowledge to implement, it is potentially computationally wasteful. In our case, this problem is made more severe by the presence of a dense Hessian. \cite{inlacoxph} use the INLA software which uses a custom adaptive quadrature rule which avoids the need for the user to choose points, however may still result in a large number of points being used.

We now describe how we mitigate the computational challenges in fitting the approximations (\ref{eqn:integration}) to the Cox PH model with partial likelihood.

\section{Methods}\label{sec:method}

\subsection{Sparse Precision Matrix and Dense Hessian}

For fixed design matrices $A$, $B$ and $X$, it is convenient to write $\eta$ and $\Delta$ as:
\begin{equation}\begin{aligned}
\eta &= A\Gamma + B\xi + X\beta \\
\Delta &= D\eta + \epsilon
\end{aligned}\end{equation}
where $\epsilon \sim \text{N}\left( 0,\tau^{-1}I_{N}\right)$ and $D$ is an $(N -1) \times N $-dimensional matrix of rank $N -1$. The partial likelihood (\ref{eqn:partial}) depends on $\eta$ only through $\Delta$, which explains why a global intercept $\beta_{0}$ is not estimable when using partial likelihood. 

The precision matrix is given by
\begin{equation}\label{eqn:precmat}
Q_{\theta} = \tau\begin{pmatrix}
I & -DA & -DB & - DX \\
- A^{T}D^{T} & \frac{1}{\tau}\Sigma_{\Gamma}^{-1} +  A^{T}D^{T}DA &  A^{T}D^{T}DB &  A^{T}D^{T}DX \\
- B^{T}D^{T} &  B^{T}D^{T}DA & \frac{1}{\tau}\Sigma_{\xi}^{-1} +  B^{T}D^{T}DB & B^{T}D^{T}DX \\
- X^{T}D^{T} &  X^{T}D^{T}DA & X^{T}D^{T}DB & \frac{1}{\tau}\Sigma_{\beta}^{-1} +  X^{T}D^{T}DX \\
\end{pmatrix}
\end{equation}
Expressions for $D$ and the derivation of this precision matrix are given in Appendix A. Contrasting (\ref{eqn:precmat}) with the precision matrix given by \citet{casecross}, we observe that ours is more sparse, owing to the addition of the noise term to $\Delta$ where they instead add it to $\eta$.

The Hessian matrix $H_{\theta}(W)$ has the form $H_{\theta}(W) = Q_{\theta} + C(W)$ where
\begin{equation*}
C(W) = -\frac{\partial^{2}}{\partial W\partial W^{T}}\ell(\Delta) = -\begin{pmatrix}
\frac{\partial^{2}\ell(\Delta;y)}{\partial\Delta\partial\Delta^{T}} & 0 & 0 \\
0 & 0 & 0 \\
0 & 0 & 0 \\
\end{pmatrix}
\end{equation*}
where the upper left block $\frac{\partial^{2}\ell(\Delta;y)}{\partial\Delta\partial\Delta^{T}}$ is a dense matrix.

Because the partial likelihood in the Cox PH model takes the form (\ref{eqn:whyINLAfail1}), $C(W)$ has a dense structure. In contrast, \cite{inla} assume that the likelihood takes the form (\ref{eqn:whyINLAfail2}) which enforces the constraint that $C(W)$ is diagonal and hence their method cannot fit the Cox PH model with partial likelihood. \cite{casecross} relax this assumption to allow $C(W)$ to have a block-diagonal structure, introducing increased computational burden when fitting the approximations. This burden becomes too limiting when $C(W)$ is dense, and an alternate method of fitting the approximations is required.

\subsection{Optimization method}\label{subsec:opt}

The objective function (\ref{eqn:modeandhessian}) is convex and high-dimensional, and hence \emph{trust region methods} \citep{trustoptim} are well-suited to the task of optimizing it. However, evaluating the Hessian at each iteration is computationally expensive. To compute the conditional mode $\hat{W}(\theta)$ when the Hessian is large, we use Symmetric Rank-1 (SR1) quasi-Newton updates within trust region optimization. The SR1 updates replace this matrix with a low-rank approximation at each iteration and hence do not require evaluation or storage of it during optimization. While this can increase the number of iterations, the computation loads brought by the dense Hessian matrix are greatly reduced and we are able to perform the optimization (\ref{eqn:modeandhessian}) when $H_{\theta}(W)$ is dense. Note that in cases where the Hessian is small and storing it in memory is convenient, we use ordinary dense-matrix trust region optimization \citep{trust} which may improve the speed of the procedure.

The quantities $\widehat{W}_{\theta}$ and $H_{\theta}(\widehat{W}_{\theta})$ are used to compute the approximations (\ref{eqn:integration}) and the associated marginal moments and quantiles, and hence the dense $H_{\theta}(\widehat{W}_{\theta})$ does need to be stored in memory. However, the total number of Hessian matrices that needs to be evaluated and stored equals to the number of quadrature points being used for the approximations (\ref{eqn:integration}). Our use of adaptive quadrature gives accurate results using fewer points than the strategies employed by \cite{casecross} and \cite{inlacoxph}. We elaborate on this now.

\subsection{Adaptive Quadrature}

To mitigate the computational challenges associated with applying a manual quadrature rule for (\ref{eqn:integration}), we implement Adaptive Gauss-Hermite Quadrature (AGHQ). This technique has been motivated as a useful tool for Bayesian inference \citep{nayloradaptive} and work has been done to show that it is very accurate when using only a very small number of quadrature points \citep{adaptive_GH_1994,adaptive_GH_2020}, for example attaining $O(n^{-1})$ asymptotic accuracy with $K = 3$ and $O(n^{-2})$ with $K = 5$. The use of only $K = 3$ or $K = 5$ quadrature points means only this many dense Hessians need to be stored in memory, a marked improvement over \cite{casecross} that is necessary to extend their method to work with the partial likelihood of the Cox PH model.

Computing the AGHQ rule requires computation of the mode of the Laplace approximation:
\begin{equation}\begin{aligned}
\widehat{\theta} &= \text{argmax}\log\widetilde{\pi}_{LA}(\theta|y),
\end{aligned}\end{equation}
as well as its Hessian matrix and its Cholesky. These matrix quantities are straightforward to obtain as $\theta$ is low-dimensional. For the optimization, we follow \citet{inla} and use numerical derivatives and a BFGS algorithm which limits the number of derivatives which must be computed. Computations make use of the \texttt{aghq} package \citep{stringeraghqpackage}. While this optimization requires repeated optimizations of $\pi(W|\theta,y)$, these are each fast owing to our use of SR1 and/or dense trust region optimization (\S\ref{subsec:opt}). The Hessian does not need to be stored in memory during the AGHQ optimization.

\section{Examples}\label{sec:example}

In this section we present a simulation study and two data analysis examples. Code is available in the online supplementary materials.

\subsection{Simulation study}\label{subsec:sim}

To illustrate the accuracy of our proposed approach, we simulate six different datasets, under three baseline hazard functions, and two true covariate effect functions. The three baseline hazard functions are respectively a hazard function constant at 0.03, an regular hazard function that is periodic and continuous, and a complicated hazard function that has three different types of oscillating behaviours and two discontinuities. The two true covariate effect functions are a simple function $\gamma(u) = \text{log}((u + 1)^2) - 1$ for $u \in (0,6)$, and a complicated function $\gamma(u) = 1.5 \text{sin}(0.8 u)$ for $ u \in (-5,5)$.

In all the six simulation settings, we  generated $N = 500$ data points from the corresponding model $h(t) = h_{0}(t) \text{exp}(\gamma(u))$, and randomly censored $10\%$ of the data points for each setting. The covariate $u$ is generated uniformly from its corresponding domain, and discretized into 50 disjoint, evenly-spaced bins. For the single variance parameter $\sigma$ that controls the smoothness of the inferred $\gamma$, we use an $\text{Exponential}(\lambda)$ prior with $\lambda$ chosen such that $\mathbb{P}\left( \sigma > 2.5\right) = 0.5$ for all settings. This is a penalized complexity prior of \cite{pcprior}. 

We fit a RW2 model using our proposed approach to infer the true smoothing function $\gamma$, with inferred $\gamma$ constrained to be zero at the median of the observed covariate values of each dataset. The resulting plots are shown in Figure 1, where the covariate effect functions are exponentiated to represent the relative risk function, and the variance parameter $\theta$ is transformed back to standard deviation $\sigma$. For the adaptive quadrature we used in our inference, the number of grid points for variance parameter to is set to be $K = 3$. We follow \citet{inla} and the associated INLA software and make use of a polynomial interpolant when making plots of the posterior distributions for $\theta$ and transformations of it.

According to Figures \ref{fig:simulation1} and \ref{fig:simulation2}, it can be seen that our proposed method gives accurate point estimation for the true risk functions regardless of the true baseline hazard function, even with a sample size that is comparatively small. In all of these six simulation examples, our 95 percent credible interval contains the true risk function throughout the domain, with an reasonable interval width at most of the regions in the domain. 

\subsection{Leukaemia Data}\label{subsec:leuk}

We implemented our proposed procedure to fit a semi-parametric Cox PH model to the Leukaemia data set analysed by \cite{inlacoxph}. The dataset contains information from 1043 independent adult leukaemia patients, with $16\%$ of observations right-censored. We are interested in quantifying the relationship between survival rate of leukaemia patients with the age of the patient, the count of white blood cells at diagnosis (wbc), the Townsend deprivation index (tpi) corresponding to the patient's location, and sex of the patient.

The effects of age, sex and white blood cell count were modelled linearly. The deprivation index was discretized into 50 equally spaced bins and modelled as a semi-parametric effect. Prior distributions $\beta \stackrel{iid}{\sim} \mathcal{N}(0, 0.001^{-1})$, were used for the regression coefficients. The semi-parametric effects $\Gamma = \{\Gamma_{1}, \cdots, \Gamma_{50}\}$ were modelled using the RW2 model of \S\ref{subsec:smooth} with the reference constraint $\gamma(0) = 0$. The single variance parameter $\sigma_{1}$ was given an $\text{Exponential}$ prior with a prior median of 2. For the adaptive quadrature we used in our inference, the number of grid points for variance parameter to is set to be $K = 3$.

Figure \ref{fig:leuk} shows the posterior result of the exponentiated covariate effect of tpi and the corresponding standard deviation $\sigma$. Our inferred covariate effect of tpi is similar to the result reported by \cite{inlacoxph}, where the risk of death initially grows with the value of tpi with diminishing rate and eventually begins to decrease after tpi reaches around 5. However, the approach utilized by \cite{inlacoxph} relies on the use of full likelihood function with baseline hazard function modelled semi-parametrically. Our approach implements the approximate Bayesian inference using the partial likelihood, hence has no assumption on the form of the baseline hazard function. 


\subsection{Kidney Catheter Data}\label{subsec:kidney}

\cite{freqPL} analysed a Kidney Catheter dataset using their proposed penalized partial likelihood method. The Kidney Catheter dataset contains 76 times to infection at the point of insertion of a catheter, for $n = 38$ patients. Each patient $i=1,\ldots,n$ forms a group, and the survival times are the time to infection of each patient's $n_{i} = 2$ kidneys. An observation for the survival time of a kidney is censored if the catheter is removed for reasons other than an infection. 

The method of \cite{freqPL} allows the inclusion of subject-level random effect (frailty) in the model to account for the within subject correlations. They suggested two different ways to estimate covariance matrix of their estimators. However the penalized partial likelihood method that they proposed is a frequentist method, where the variance of random effects is treated as a fixed value estimated through either MLE or REML. As they have mentioned in the paper, ignoring the uncertainty with estimating the variance parameter in this way can severely underestimate the standard errors. For this Kidney Catheter dataset, they found that their methods tend to underestimate the true standard errors through some bootstrap experiments \citep{freqPL}. They suggest an alternative variance estimator which may account for this, but do not offer clear advice on which estimator to choose in practice.

As a comparison, we also use our procedure to fit a Cox PH model to these grouped data, providing full posterior uncertainty over the between-subject standard deviation. We associate survival times with covariates sex, age, and indicator of one of four types of pre-existing disease each patient may have. Subject-specific intercepts $\xi_{i}\overset{iid}{\sim}\text{N}(0,\sigma^{2}_{\xi})$ are included to account for correlation between kidneys from the same subject. We use an $\text{Exponential}$ prior distribution for $\sigma_{\xi}$ with median 2. For the adaptive quadrature we used in our inference, the number of grid points for variance parameter to is set to be 3. For both implementations of the proposed method and the penalized partial likelihood method, ties are handled using the method of \cite{Breslow}.

Table \ref{table:KidneyFixed} shows the results of our procedure compared to that obtained using the penalized partial likelihood method of \cite{freqPL}. In the table, SD1 and SD2 are the two different methods \cite{freqepsPL} suggested to estimate the standard errors. Our posterior means for the linear covariate effects are comparable to the frequentist estimates. However, our posterior standard deviations of linear effects are larger than both of the two versions of estimated standard errors from the penalized likelihood method. This is a result of our taking into account the uncertainty with estimating the variance parameter, and we see that our estimates are closer to their more conservative one. In contrast, however, we offer an unambiguous choice of estimator to use, which is a practical advantage. Further, as shown in Figure \ref{fig:BetweenSubjectSD}, our method provides full posterior uncertainty for $\sigma_{\xi}$ while the penalized partial likelihood approach only provides a point estimate.

\section{Discussion}\label{sec:discussion}

The methodology we proposed in this paper provides a flexible way to carry out Bayesian inference for Cox proportional hazard models with partial likelihood, that accommodates the inference for semi-parametric covariate effects and correlated survival times. The use of partial likelihood does not require any assumption on the baseline hazard function, which is an advantage over existing approaches for Bayesian inference in this model. We have demonstrated the accuracy of our new approach through some simulation studies, even when the sample size is small, and we have implemented our proposed method to analyse two classical datasets in survival analysis. Our proposed method is an appealing option to adopt for the analysis of time-to-event data.

One limitation of our proposed methodology is the manner in which it scales with the sample size $N$. Since the Hessian matrix in our methodology is fully dense, its number of non-zero entries increases as $O(N^{2})$. The scalability of our procedure is limited by the need to store this matrix in memory. We avoid the computation of this Hessian matrix during the optimization step by using a quasi-Newton method, however the true Hessian matrix is still required to be evaluated and stored at the maximum to compute the posterior approximations that we use. The computational requirements will increase with the dimension of $\theta$ as well, a limitation that our method shares with \citet{inla} and \cite{casecross}.

The framework of this proposed methodology can be extended to fit more complex models, by modifying the covariance structure of the covariate with semi-parametric effect. Temporally- and spatially-correlated survival data may be analysed through a similar procedure. Because we accommodate the dense Hessian matrix of the log-likelihood, our approach could be extended to approximate Bayesian inference for other models with a dense Hessian matrix. We leave such extensions to future work.

\section*{Data Availability Statement}
The simulated data of example 3.1 are available in the supplementary material with this paper. Data for example 3.2 were obtained from R package "INLA" \citep{inla} and are freely available.
Data for example 3.3 were obtained from R package "survival" \citep{survival-package} and are freely available. 

\appendix

\section{Derivation of Precision Matrix}

In this section we give a brief derivation of the precision matrix $Q_{\theta}$ from Equation (\ref{eqn:precmat}). The derivation is similar to that of \cite{casecross} (Web Appendix C), with a different differencing matrix. The differencing matrix $D$ is:
\begin{equation}\begin{aligned}\label{eqn:D2}
D &= \begin{pmatrix}
1 & -1 & 0 & & 0 \\
1 & 0 & -1 & & 0 \\
  &    &  & \ddots  &   \\
1 &    &       & 0 & -1 \\
\end{pmatrix}
\end{aligned}\end{equation}
As described in \S\ref{sec:method} ,our model specifies:
\begin{equation*}
\Gamma|\theta \sim \text{Normal}\left( 0,\Sigma_{\Gamma}\right); \ \xi|\theta \sim \text{Normal}\left( 0,\Sigma_{\xi}\right); \ \beta \sim \text{Normal}\left( 0,\Sigma_{\beta}\right); \ \epsilon \sim \text{Normal}\left( 0,\tau^{-1}I\right)
\end{equation*}
all independent of each other unless otherwise specified. The vector of additive linear predictors can be written as $\eta = A\Gamma + B\xi + X\beta + \epsilon$ and $\Delta = D\eta$ where $D$ is defined through Equation (\ref{eqn:D2}). This gives a joint Gaussian distribution for $W|\theta$ as:
\begin{equation*}
W|\theta = \begin{pmatrix} \Delta \\ \Gamma \\ \xi \\\beta \end{pmatrix} = \begin{pmatrix} DA & DB & DX & I \\ I & 0 & 0 & 0 \\ 0 & I & 0 & 0 \\ 0 & 0 & I & 0 \\ \end{pmatrix}\begin{pmatrix}\Gamma\\ \xi \\ \beta \\ \epsilon \end{pmatrix} 
\sim \text{Normal}\left( 0,\Sigma\right)
\end{equation*}
where
\begin{equation*}
\Sigma = \begin{pmatrix}
DA\Sigma_{\Gamma}A^{T}D^{T} + DB\Sigma_{\xi}B^{T}D^{T} + DX\Sigma_{\beta}X^{T}D^{T} + \tau^{-1}I & DA\Sigma_{\Gamma} & DB\Sigma_{\xi} & DX\Sigma_{\beta} \\
\Sigma_{\Gamma}D^{T}A^{T} & \Sigma_{\Gamma} & 0 & 0 \\
\Sigma_{\xi}D^{T}B^{T} & 0 & \Sigma_{\xi} & 0 \\
\Sigma_{\beta}D^{T}X^{T} & 0 & 0 & \Sigma_{\beta} \\
\end{pmatrix}
\end{equation*}
The precision matrix $Q(\theta) = \Sigma^{-1}$ is obtained through direct inversion.


\nocite{*}% Show all bib entries - both cited and uncited; comment this line to view only cited bib entries;
\bibliographystyle{ba}
\bibliography{myrefs}%

\clearpage




\end{document}

